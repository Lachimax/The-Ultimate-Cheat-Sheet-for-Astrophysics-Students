\appendix

\chapter{Glossary of Symbols}
	
		\begin{description}		
        	\item $\vec{x}$ : A pronumeral with an arrow above it represents a vector quantity. 		
			\item $x$ : When that same pronumeral is seen without the arrow, it represents the \textit{magnitude} of that vector.	
			\item $\hat{x}$ : A hat indicates the \textit{unit vector}; that is, a vector with magnitude 1. When replacing the arrow on a previously used vector, it indicates that it points in the direction of that vector.	
			\item $\bar{x}$ : A bar indicates that this is the average value over several items.
			\item $\dot{x}:$ A dot indicates the first derivative of the value.		
			\item $\ddot{x}:$ Two dots is the second derivative, and so on.
		
		These things stack.
        
		\end{description}
		
		
		
	
		\begin{itemize}
			
				\item $\iff$
				\begin{itemize}
					\item\textbf{Maths:} If and only if
				\end{itemize}
			
				\item $\parallel$
				\begin{itemize}
					\item \textbf{Maths:} parallel, parallel to
				\end{itemize}
				
				\item $\perp$
				\begin{itemize}
					\item \textbf{Maths:} perpendicular, perpendicular to
				\end{itemize}
				
				\item $\odot$
				\begin{itemize}
					\item \textbf{Astronomy:} (as a subscript) pertaining to the Sun.
				\end{itemize}
				
				\item $*$
				\begin{itemize}
					\item \textbf{Maths:} (as a superscript) complex conjugate
				\end{itemize}
				
				\item $\dagger$
				\begin{itemize}
					\item \textbf{Maths:} (as a superscript) Hermitian adjoint of a matrix
				\end{itemize}
				
				\item $-1$
				\begin{itemize}
					\item \textbf{Maths:} (as a superscript) inverse function, or to the power of -1; these two are often confused and must be separated based on context.
				\end{itemize}
			
				\item \textit{A}
				\begin{itemize}
					\item\textbf{Maths:} area
					\item\textbf{Physics:} area, especially cross-sectional
				\end{itemize}
				
				\item \textit{a}
				\begin{itemize}
					\item\textbf{Maths (Geometry):} The length of the long axis of an ellipse.
					\item\textbf{Physics:} acceleration
					\begin{itemize}
						\item arbitrary position (accompanied by b)
					\end{itemize}
				\end{itemize}
				
				\item \textit{B}
				\begin{itemize}
					\item\textbf{Physics:} magnetic field
				\end{itemize}
				
				\item $b$
				\begin{itemize}
					\item\textbf{Physics:} arbitrary position (accompanied by a)
					\item\textbf{Maths (Geometry):} The length of the shortest axis of an ellipse.
				\end{itemize}
				
				\item $C$
				\begin{itemize}
					\item\textbf{Unit:} Coulomb (charge)
				\end{itemize}
				
				\item $c$
				\begin{itemize}
					\item\textbf{Physics:} speed of light
					\begin{itemize}
						\item (as a subscript) centripetal
					\end{itemize}
				\end{itemize}
				
				\item $D$
				\begin{itemize}
					\item\textbf{Maths:} in linear algebra, the diagonalisation of a matrix; has the eigenvalues as its entries.
					\item\textbf{Astronomy (instrumentation):} Lens diameter
				\end{itemize}
				
				\item $D_T$
				\begin{itemize}
					\item Telescope diameter
				\end{itemize}
				
				\item $d$
				\begin{itemize}
					\item\textbf{Maths (Calculus):} preceding a quantity, it represents an infinitesemal difference in that quantity.
					\item\textbf{Maths (Linear Algebra):} The dimension of a space
					\item\textbf{Physics:} (as subscript) displaced
					
				\end{itemize}
				
				\item \textit{E}
				\begin{itemize}
					\item\textbf{Physics:} energy
					\item\textbf{Astronomy:} (as a subscript) Pertaining to the Earth
				\end{itemize}
				
				\item \textit{e}
				\begin{itemize}
					\item\textbf{Physics:} the elementary charge, charge of a proton
				\end{itemize}
				
				\item \textit{F}
				\begin{itemize}
					\item\textbf{Physics:} force, net force
					\item\textbf{Astronomy:} flux
				\end{itemize}
				
				\item $F_0$
				\begin{itemize}
					\item\textbf{Astronomy:} reference flux
				\end{itemize}
				
				\item $f$
				\begin{itemize}
					\item\textbf{Physics:} frequency (also $nu$)
					\item\textbf{Physics (optical):} Focal length
				\end{itemize}
				
				\item $f_{sys}$
				\begin{itemize}
					\item\textbf{Astronomy (instrumentation):} Focal length of the system.
				\end{itemize}
				
				\item $f_\#$
				\begin{itemize}
					\item\textbf{Astronomy (instrumentation):} Focal ratio or focal number
				\end{itemize}
				
				\item $G$
				\begin{itemize}
					\item\textbf{Physics:} universal gravitational constant
					\begin{itemize}
						\item (as a subscript) to do with gravity
					\end{itemize}
				\end{itemize}
				
				\item \textit{g}
				\begin{itemize}
					\item\textbf{Unit:} gram
					\item\textbf{Physics:} acceleration due to gravity; usually 9.8 $ms^{-2}$
				\end{itemize}
				
				\item \textit{H}
				\begin{itemize}
					\item\textbf{Maths:} a Hermitian matrix
					\item\textbf{Chemistry:} Hydrogen
					\item\textbf{Astronomy:} Mass of hydrogen in an object
				\end{itemize}
				
				\item $h$
				\begin{itemize}
					\item\textbf{Physics:} height
					\item\textbf{Astronomy:} height of the galaxy at a given point.
				\end{itemize}
				
				\item $I$
				\begin{itemize}
					\item\textbf{Physics:} electrical current
					\item\textbf{Maths:} The identity matrix
				\end{itemize}
				
				\item $i$
				\begin{itemize}
					\item \textbf{Maths:} imaginary unit, $\sqrt{-1}$
					\item\textbf{Physics:} electrical current per unit charge
				\end{itemize}
				
				\item $j$
				\begin{itemize}
					\item\textbf{Maths (linear algebra):} The row of a matrix entry
				\end{itemize}
				
				\item $k$
				\begin{itemize}
					\item\textbf{Maths (linear algebra):} The column of a matrix entry
				\end{itemize}
				
				\item $k_B$
				\begin{itemize}
					\item\textbf{Physics:} Boltzmann constant
					\item
				\end{itemize}
				
				\item $k_e$
				\begin{itemize}
					\item\textbf{Physics (Electromagnetism):} Coulomb's constant
					\item
				\end{itemize}
				
				\item $k_s$
				\begin{itemize}
					\item\textbf{Physics:} Spring stiffness constant
					\item
				\end{itemize}
				
				\item $L$
				\begin{itemize}
					\item\textbf{Physics:} length
					\begin{itemize}
						\item 
					\end{itemize}
				\end{itemize}
				
				\item $l$
				\begin{itemize}
					\item\textbf{Physics:} length
					\begin{itemize}
						\item path
					\end{itemize}
				\end{itemize}
				
				\item $\ell$
				\begin{itemize}
					\item\textbf{Physics:} Lagrangian
				\end{itemize}
				
				\item $M$
				\begin{itemize}
					\item\textbf{Physics:} mass, especially the larger of two or as a constant
					\item\textbf{Astronomy:} absolute magnitude
				\end{itemize}
				
				\item $m$
				\begin{itemize}
					\item\textbf{Unit:} metre (length)
					\item\textbf{Maths:} the number of rows in a matrix
					\item\textbf{Physics:} mass, especially the smaller of two or as a variable
					\item\textbf{Astronomy:} apparent magnitude
				\end{itemize}
				
				\item $m_0$
				\begin{itemize}
					\item \textbf{Astronomy:} reference magnitude
				\end{itemize}
				
				\item $n$
				\begin{itemize}
					\item\textbf{Maths:} the number of columns in a matrix
					\begin{itemize}
						\item the number of rows and the number of columns in a square matrix
					\end{itemize}
					\item\textbf{Physics:} number density; ie, the number per unit volume or area
				\end{itemize}
				
				\item $N$
				\begin{itemize}
					\item\textbf{Astronomy (instrumentation):} Focal ratio
				\end{itemize}
				
				\item $P$
				\begin{itemize}
					\item\textbf{Maths:} the eigenvector matrix
					\begin{itemize}
						\item a projector
						\item probability
					\end{itemize}
					\item\textbf{Physics:} power
					\begin{itemize}
                    	\item pressure
						\item probability
					\end{itemize}
				\end{itemize}
				
				\item $p$
				\begin{itemize}
					\item\textbf{Physics:} momentum
                    \begin{itemize}
                    	\item pressure
                    \end{itemize}
					\item\textbf{Astronomy (Instrumentation):} Pixel length
				\end{itemize}
				
				\item \textit{R}
				\begin{itemize}
					\item\textbf{Maths:} radius, especially fixed
					\item\textbf{Physics:} resistance
					\item\textbf{Astronomy (Instrumentation):} The radius of curvature of a lens
				\end{itemize}
				
				\item $R_1$
				\begin{itemize}
					\item\textbf{Astronomy (Instrumentation):} The radius of curvature of the object-facing side of a lens.
				\end{itemize}
				
				\item $R_2$
				\begin{itemize}
					\item\textbf{Astronomy (Instrumentation):} The radius of curvature of the image-facing side of a lens.
				\end{itemize}
				
				\item $R_S$
				\begin{itemize}
					\item \textbf{Physics (Relativity):}Schwarszchild radius
				\end{itemize}
				
				\item $r$
				\begin{itemize}
					\item\textbf{Maths:} radius, especially as a variable
					\begin{itemize}
						\item distance
						\item distance from origin in polar or spherical coordinates
						\item distance from the z-axis in cylindrical coordinates (also $\rho$)
					\end{itemize}
					\item\textbf{Physics:} position
					\begin{itemize}
						\item radius
						\item distance 
					\end{itemize}
				\end{itemize}
				
				\item$r_0$
				\begin{itemize}
					\item \textbf{Astronomy (Instrumentation):} The Fried parameter, or atmospheric coherence length, used as a measurement of atmospheric turbulence. 
				\end{itemize}
								
				\item$r'$
				\begin{itemize}
					\item \textbf{Physics:} position of the source of a field
				\end{itemize}
				
				\item \textit{S}
				\begin{itemize}
					\item\textbf{Physics (Relativity):} A frame of reference
					\begin{itemize}
						\item (as a subscript) Schwarszchild
					\end{itemize}
				\end{itemize}
				
				\item \textit{s}
				\begin{itemize}
					\item\textbf{Unit:} second
					\item\textbf{Physics:} separation
					\begin{itemize}
						\item (as a subscript) Relating to springs.
					\end{itemize}
				\end{itemize}
				
				\item \textit{T}
				\begin{itemize}
					\item \textbf{Maths:} (as a superscript) transpose of a matrix
					\item\textbf{Physics:} period
					\begin{itemize}
						\item temperature                      
					\end{itemize}
				\end{itemize}
				
				\item \textit{t}
				\begin{itemize}
					\item\textbf{Physics:} time
				\end{itemize}
				
				\item \textit{U}
				\begin{itemize}
					\item\textbf{Physics:} potential energy
					\item\textbf{Maths:} a unitary matrix
				\end{itemize}
				
				\item \textit{u}
				\begin{itemize}
					\item\textbf{Physics:} initial velocity
				\end{itemize}
				
				\item \textit{V}
				\begin{itemize}
					\item\textbf{Physics:} voltage
					\begin{itemize}
						\item electric potential
						\item volume
					\end{itemize}
				\end{itemize}
				
				\item $v$
				\begin{itemize}
					\item\textbf{Physics:} velocity
				\end{itemize}
				
				\item $w$
				\begin{itemize}
					\item\textbf{Maths:} width
				\end{itemize}
				
				\item $w_D$
				\begin{itemize}
					\item\textbf{Astronomy (instrumentation):} width of detector
				\end{itemize}
				
				\item $X$
				\begin{itemize}
					\item\textbf{Physics:} Pauli operator
					\begin{itemize}
						\item 
					\end{itemize}
				\end{itemize}
				
				\item \textit{x}
				\begin{itemize}
					\item\textbf{Maths:} a variable, especially in a function.
					\begin{itemize}
						\item the horizontal axis in Cartesian coordinates.
					\end{itemize}
					\item \textbf{Physics:} position
				\end{itemize}
				
				\item $x_0$
				\begin{itemize}
					\item\textbf{Maths:} the initial value of variable $x$.
				\end{itemize}
				
				\item \textit{Y}
				\begin{itemize}
					\item\textbf{Physics:} A Pauli operator
					\begin{itemize}
						\item 
					\end{itemize}
				\end{itemize}
				
				\item \textit{y}
				\begin{itemize}
					\item\textbf{Maths:} an axis in Cartesian coordinates.
					\item \textbf{Physics:} position, usually vertical.
				\end{itemize}
				
				\item \textit{Z}
				\begin{itemize}
					\item\textbf{Physics:} A Pauli operator
					\begin{itemize}
						\item 
					\end{itemize}
				\end{itemize}
				
				\item \textit{z}
				\begin{itemize}
					\item\textbf{Maths:} the vertical axis in Cartesian coordinates.
					\begin{itemize}
						\item A complex number
					\end{itemize}
					\item \textbf{Physics:} position
				\end{itemize}
				
				\item $z_0$
				\begin{itemize}
					\item\textbf{Maths:} the initial value of variable $z$.
				\end{itemize}
								
				\item $\alpha$
				
				\item $\beta$
				
				\item $\Gamma$
				
				\item $\gamma$
				\begin{itemize}
					\item \textbf{Physics:} gamma factor
				\end{itemize}
				
				\item $\Delta$
				\begin{itemize}
					\item \textbf{Maths:} difference between two quantities
					\item \textbf{Physics:} difference between two quantities
					\begin{itemize}
						\item \textbf{Standard deviation}
					\end{itemize}
				\end{itemize}
				
				\item $\delta$
				\begin{itemize}
					\item 
				\end{itemize}
				
				\item $\partial$
				\begin{itemize}
					\item \item\textbf{Maths (Calculus):} denotes the patial derivative
				\end{itemize}
								
				\item $\epsilon$
				\begin{itemize}
					\item 
				\end{itemize}	
								
				\item $\varepsilon$
				\begin{itemize}
					\item \textbf{Physics:} electromotive force (force per unit charge)
				\end{itemize}
				
				\item $\varepsilon_d$
				\begin{itemize}
					\item \textbf{Astronomy (instrumentation):} Diffraction limit to angular resolution.
				\end{itemize}
				
				\item $\varepsilon_s$
				\begin{itemize}
					\item \textbf{Astronomy (instrumentation):} Seeing limit to angular resolution.
				\end{itemize}
				
				\item $\zeta$
				\begin{itemize}
					\item 
				\end{itemize}
				
				\item $\eta$
				\begin{itemize}
					\item 
				\end{itemize}
				
				\item $\Theta$
				\begin{itemize}
					\item 
				\end{itemize}
				
				\item $\theta$
				\begin{itemize}
					\item \textbf{Maths:} An angle
				\end{itemize}
				
				\item $\vartheta$
				\begin{itemize}
					\item 
				\end{itemize}
				
				\item $\iota$
				\begin{itemize}
					\item 
				\end{itemize}
				
				\item $\kappa$
				\begin{itemize}
					\item 
				\end{itemize}
				
				\item $\Lambda$
				\begin{itemize}
					\item 
				\end{itemize}
				
				\item $\lambda$
				\begin{itemize}
					\item \textbf{Maths:} eigenvalue
					\item \textbf{Physics:} wavelength
				\end{itemize}
				
				\item $\mu$
				\begin{itemize}
					\item \textbf{Physics:} mass density
				\end{itemize}
				
				\item $\nu$
				\begin{itemize}
					\item \textbf{Physics:} frequency (also $f$)
				\end{itemize}
				
				\item $\Xi$
				\begin{itemize}
					\item 
				\end{itemize}
				
				\item $\xi$
				\begin{itemize}
					\item 
				\end{itemize}
				
				\item $\Pi$
				\begin{itemize}
					\item 
				\end{itemize}
				
				\item $\pi$
				\begin{itemize}
					\item\textbf{Maths:} the ratio of a circle's circumference to its diameter.
				\end{itemize}
				
				\item $\rho$
				\begin{itemize}
					\item \textbf{Physics:} density
					\begin{itemize}
						\item resistivity
					\end{itemize}
				\end{itemize}
				
				\item $\varrho$
				\begin{itemize}
					\item 
				\end{itemize}
				
				\item $\Sigma$
				\begin{itemize}
					\item 
				\end{itemize}
				
				\item $\sigma$
				\begin{itemize}
					\item \textbf{Physics:} conductivity
					\begin{itemize}
						\item Pauli operator
					\end{itemize}
					\item\textbf{Astronomy (Instrumentation:)} instrument error
				\end{itemize}
				
				\item $\sigma_{fit}$
				\begin{itemize}
					\item\textbf{Astronomy (Instrumentation:)} fitting error
				\end{itemize}
				
				\item $\tau$
				\begin{itemize}
					\item \textbf{Physics:} in relativity, proper time
				\end{itemize}
				
				\item $\upsilon$
				\begin{itemize}
					\item 
				\end{itemize}
				
				\item $\Upsilon$
				\begin{itemize}
					\item 
				\end{itemize}
				
				\item $\Phi$
				\begin{itemize}
					\item \textbf{Physics:} Gravitational potential (GPE per unit mass)
				\end{itemize}
				
				\item $\phi$
				\begin{itemize}
					\item 
				\end{itemize}
				
				\item $\varphi$
				\begin{itemize}
					\item 
				\end{itemize}
				
				\item $\Psi$
				\begin{itemize}
					\item 
				\end{itemize}
				
				\item $\psi$
				\begin{itemize}
					\item \textbf{Physics (Quantum):} a quantum state
				\end{itemize}
				
				\item $\omega$
				\begin{itemize}
					\item 
				\end{itemize}
				
				\item $\Omega$
				\begin{itemize}
					\item 
				\end{itemize}
				
				
		\end{itemize}
        
\chapter{Mathematical Stuff}

	\section{Set Notation}
		
		\begin{itemize}
			\item $\mathbf{N}$ - The set of natural numbers - i.e. integers greater than or equal to zero.
			\item $\mathbf{Z}$ - The set of integers.
			\item $\mathbf{Q}$ - The set of rational numbers - i.e. numbers expressible as integer fractions.
			\item $\mathbf{R}$ - The set of real numbers.
			\item $\mathbf{C}$ - The set of complex numbers (including real numbers).
			\item $\mathbf{H}$ - A Hilbert space
			\item $\in$ - element of
			\item $\forall$ - for all
			\item $\cap$ - intersection of
			\item $\cup$ - union of
		\end{itemize}
		
	\section{Maths Dictionary}
	
		\begin{description}			
			\item[Basis:] A set of linearly independent vectors which span a vector space; that is, any vector in that space can be written as a linear combination of the basis vectors.
			
			\item[Bijective:] A function that is both injective and surjective.
			
			\item[Co-domain:] The set of numbers which contains, but may not be limited to, the range of a function. That is, all of a function's outputs are in its co-domain.
			
			\item[Commute:] Two matrices commute if $[A,B] = 0$; ie, $AB = BA$.
			
			\item[Complex transpose:] Hermitian adjoint.
			
			\item[Convex set:] In a convex set, all points on a line connecting any two elements are also in the set.
			
			\item[Complete: ] If every Cauchy sequence converges to a vector in a vector space, that space is complete. All finite complex vector spaces are complete.
			
			\item [Closure (of a set):] Both the boundary and interior points of the set.
			
			\item [Column space (of a matrix):] The subspace of $\mathbf{R}^n$ spanned by the columns of the matrix. That is, if a vector is a linear combination of the columns of the matrix, it is in the column space of that matrix.
			
			\item [Consistent system of equations:] If a system has one or more solutions, it is consistent. If it has no solutions, it is inconsistent.
			
			\item[Diagonal Matrix:] A matrix with non-zero entries only in its diagonal.
			
			\item[Differentiability (of a multivariable function):] If the partial derivatives of a function $f(\vec{x})$ exist and are continuous at $\vec{x} = \vec{a}$, then $f(x)$ is differentiable at $\vec{a}$
			
			\item[Dimension:] The minimum number of vectors required to span a vector space; or, the definitive number of vectors in a basis of that space.
			
			\item[Domain (of a function):] The set of numbers from which a function's inputs are chosen.
			
			\item[Finite-dimensional vector space:] A vector space spanned by a finite set of vectors; or, with a finite basis.
			
			\item[Free variable:] a variable that is unconstrained.
			
			\item[Eigenspace:] A vector space for which the eigenvectors of a matrix form a basis.
			
			\item[Hermitian:] A matrix which has its Hermitian adjoint equal to itself.
			
			\item[Hermitian adjoint:] The Hermitian adjoint $A^\dagger$ is the transpose of $A$, with each entry replaced by its complex conjugate. Also known as the complex transpose.
			
			\item[Hilbert Space:] A complex, inner-product space that is complete.
			
			\item[Image:] ?
			
			\item[Inconsistent system of equations:] A system of equations is inconsistent if it has no solutions. If it has one or more solutions, it is consistent.
			
			\item[Injective (function):] for every input, there is exactly one output. Every input is mapped to only one output by the function. (also one-to-one)
			
			\item[Inner product:] A product that satisfies conjugate symmetry, linearity in the first argument and positive-definitiveness (in quantum physics, the rule is linearity in the second argument).
			
			\item[Imaginary Matrix:] A matrix with all imaginary entries.
			
			\item[Kernel (of a transformation):] All of the points in a transformation's domain that will be sent to zero. (also nullspace)
			
			\item[Linear independence:] A vector is linearly independent of another if it cannot be written as a linear combination of the other. A set of vectors is linearly independent if all of the members are linearly independent of each other
			
			\item[Linear isomorphism:] a linear transformation that is bijective. If $T$ is an isomorphism, so is its inverse $T^{-1}$
			
			\item[Nullity:] The dimension of the nullspace of a matrix.
			
			\item[Nullspace (of a transformation):] All of the points in a transformation's domain that will be sent to zero. (also kernel)
			
			\item[Normal:] A vector is normal when it has magnitude 1.
			
			\item[Orthogonal:] When two vectors are orthogonal, their inner product is zero. See also perpendicular.
			
			\item[Onto:] The range of the function is identical to the co-domain. Also surjective.
			
			\item[Orthonormal set:] A set of vectors, especially a basis, in which all the vectors are both normal and orthogonal to each other.
			
			\item[One-to-one (function):] Injective: for every input there is exactly onbe output.
			
			\item[Parallel:] 
			
			\item[Perpendicular:] A ninety-degree angle.
			
			\item[Pivot column (of a matrix):] A column containing a pivot variable.
			
			\item[Pivot variable (of a matrix):] The first non-zero entry in a row of a matrix.
			
			\item[Positive-definite (operation or function): ] Outputs are greater than zero, except when inputs are zero, in which case equal to zero.
			
			\item[Range (of a function):] The set of a function's outputs.
			
			\item[Rank (of a matrix):] The dimension of the vector space spanned by the columns of a matrix; or, the number of leading ones (or pivot variables)in the row-reduced matrix.
			Also equal to the dimension of the image of the matrix.
			
			\item[Rank-Nullity Theorem:] The number of columns in a matrix is equal to the rank of the matrix plus the nullity of the matrix. 
			Or, the dimension of the nullspace plus the dimension of the image is the number of columns.
			
			\item[Real matrix:] A matrix with all real-valued entries.
			
			\item[Row space:] the subspace of $\mathbf{R}^n$ spanned by the row vectors of a matrix.
			
			\item[Span:] A set of vectors spans a space if any vector in the space can be written as a linear combination of the set.
			
			\item[Subspace:] A set of vectors within a space forming their own vector space. The conditions for a subspace are:
			\begin{enumerate}
				\item Must be closed under addition
				\item Must be closed under scalar multiplication
				\item Must contain the null vector
			\end{enumerate}
			
			\item[Surjective (function):] The range of the function is identical to the co-domain. Also onto.
			
			\item[Symmetric Matrix:] A square matrix which is symmetric across the diagonal.
			
			\item[Trace:] The sum of the diagonal entries of a matrix.
			
			\item[Transpose:] $A^T$, the transpose of $A$, is the reflection of $A$ across its diagonal. Only works with square matrices.
			
			\item[Vector space:] Any set of linearly independent vectors.
			
		\end{description}
\chapter{Astronomy Dictionary}
		
		\begin{description}
			
			\item[Absolute magnitude:] The apparent magnitude of an object when viewed from a distance of 10 parsecs distance (?).
			
		\end{description}
        

\section{Operations}

\begin{itemize}	
%\item $\oplus$
%\begin{itemize}
%\item The vector space $V = W\oplus \ U$ contains all linear combinations of vectors in $W$ and $U$.
%\end{itemize}

\item $x + y$
\begin{itemize}
\item Addition: Add $y$ to $x$
\end{itemize}

\item $x - y$
\begin{itemize}
\item Subtraction: Take $y$ from $x$
\end{itemize}

\item $x\times y$
\begin{itemize}
\item Multiplication: $x$ groups of size $y$.
\end{itemize}

\item $x\div y$, or $\dfrac{x}{y}$
\begin{itemize}
\item Division: Split $x$ into $y$ groups.
\end{itemize}

\item $x^n$
\begin{itemize}
\item Exponentiation: the $n$th power of $x$ - that is, $x\times x$, $n$ times.
\end{itemize}

\item $\sqrt{x}$
\begin{itemize}
\item The square root of $x$
\end{itemize}

\item $\sqrt[n]{x}$
\begin{itemize}
\item The nth root of $x$
\end{itemize}

\item $n!$
\begin{itemize}
\item Factorial of n
\item $n \times (n-1) \times (n-2) \times \ldots \times 2 \times 1$
\end{itemize}

\item $|x|$
\begin{itemize}
\item Absolute value of x
\item (for a complex number) Complex modulus of x
\end{itemize}

\item $\langle x\rangle$
\begin{itemize}
\item Average
\end{itemize}

\item $\otimes$
\begin{itemize}
\item Tensor product
\end{itemize}

\item $\oplus$
\begin{itemize}
\item Tensor addition
\end{itemize}

\item $\doteq$
\begin{itemize}
\item Equivalent to / represented as
\end{itemize}

\item $\log_a(b)$
\begin{itemize}
\item Logarithm, base a, of b.
\end{itemize}

\item ln$(a)$
\begin{itemize}
\item Logarithm, base e, of a.
\end{itemize}

\item dim$(A)$
\begin{itemize}
\item The dimension of matrix $A$.
\end{itemize}

\item ker$(T)$
\begin{itemize}
\item The nullspace/kernel of transformation $T$
\end{itemize}

\item nul$(T)$
\begin{itemize}
\item The nullity of transformation $T$
\end{itemize}

\item rk$(A)$
\begin{itemize}
\item Rank of matrix $A$
\end{itemize}

\item Tr$(A)$
\begin{itemize}
\item Trace of a matrix $A$
\end{itemize}

\item $\Im(z)$ or Im$(z)$
\begin{itemize}
\item The imaginary component of complex number $z$
\end{itemize}

\item $\Re(z)$ or Re$(z)$
\begin{itemize}
\item The real component of complex number $z$
\end{itemize}				
\end{itemize}

\subsection{What do the numbers mean}

These are based entirely on my own experience; you may or may not encounter them in a completely different order.

\begin{description}
\item[HS:] High School
\item[100:] 1st-year undergraduate
\item[200:] 2nd-year undergraduate
\item[300:] 3rd-year undergraduate
\item[700:] Masters-level
\end{description}
	