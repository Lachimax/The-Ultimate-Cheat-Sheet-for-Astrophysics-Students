\chapter{Units of Measurement}
            
            
\section{SI System}	
Universally acknowledged as the best system of units.

\subsection{Base Units}
\begin{description}

\item[Amount of Substance:] mole ($mol$)
\begin{itemize}
\item The amount of substance of a system which contains as many elementary entities as there are atoms in 0.012 $kg$ of carbon-12.
\items $= 6.022\ 140\ 857 \times 10^{23}$
\end{itemize}			

\item[Electric Current:] ampere ($A$)
\begin{itemize}
\item The constant current which, if maintained in two straight parallel conductors of infinite length, of negligible circular cross-section, and placed 1 m apart in vacuum, would produce between these conductors a force equal to $2×10^{-7} N\cdot m^{-1}$ of length.
\items $= C\cdot s^{-1} $
\end{itemize}			

\item[Length:] metre ($m$)
\begin{itemize}
\item The distance traveled by light in vacuum in $\dfrac{1}{299\ 792\ 458}\ s$
\items $=3.2808\ ft$
\end{itemize}

\item[Luminous intensity:] candela ($cd$)
\begin{itemize}
\item The luminous intensity, in a given direction, of a source that emits monochromatic radiation of frequency $5.4 \times 10^{14}$ Hz and that has a radiant intensity in that direction of $\dfrac{1}{683}$ watt per steradian.
\end{itemize}

\item[Mass:] kilogram ($kg$)
\begin{itemize}
\items $= 2.205 lb$
\end{itemize}				

\item[Thermodynamic Temperature:] kelvin ($K$)
\begin{itemize}
\item $\frac{1}{273.16}$ of the thermodynamic temperature of the triple point of water.
\items $= T [\degrees C] + 273.15$
\end{itemize}

\item[Time:] second (s)
\begin{itemize}
\item The duration of 9 192 631 770 periods of rotation corresponding to the two hyperfine levels of the ground-state of the caesium-133 atom.
\end{itemize}				
\end{description}	


\subsection{Derived Units}
\begin{description}				

\item[Angle:] radian ($rad$)
\begin{itemize}
\item A full circle divided by $2\pi.$
\items $= m\cdot m^{-1} \tab[0.5cm]= \dfrac{180}{\pi} \tab[0.5cm]= 206265\ arcsecs \tab[0.5cm]\approx 57.3 \degrees$                    
\end{itemize}				

\item[Electric Charge:] coulomb ($C$)
\begin{itemize}
\items $= A\cdot s \tab[0.5cm]= 6.242\times 10^{18}\ e$
\end{itemize}	

\item[Electrical capacitance:] farad ($F$)
\begin{itemize}
\items $=  m^{-2} \cdot kg^{-1} \cdot  s^4 \cdot A^2$
\end{itemize}				

\item[Electrical conductance:] siemens ($S$)
\begin{itemize}
\items $= A \cdot V^{-1} \tab[0.5cm]= kg^{-1} \cdot m^{-2} \cdot s^3 \cdot A^2$
\end{itemize}

\item[Electrical inductance:] henry ($H$)
\begin{itemize}
\items $= Wb\cdot A^{-1}\tab[0.5cm]= kg \cdot m^2 \cdot s^{-2} \cdot A^{-2}$
\end{itemize}	

\item[Electrical potential difference / Voltage:] volt ($V$)
\begin{itemize}
\items $= W \cdot A^{-1} \tab[0.5cm]= kg \cdot m^{2} \cdot s^{-3} \cdot A^{-1}$
\end{itemize}			

\item[Electrical resistance:] ohm ($\Omega$)
\begin{itemize}
\items $= V\cdot A^{-1} \tab[0.5cm]= kg \cdot m^{2} \cdot s^{-3} \cdot A^{-2}$
\end{itemize}

\item[Energy:] joule ($J$)
\begin{itemize}
\items $= N \cdot m \tab[0.5cm]= kg \cdot m^{2} \cdot s^{-2}$
\end{itemize}

\item[Force:] newton ($N$)
\begin{itemize}
\items $= kg \cdot m \cdot s^{-2} \tab[0.5cm] = 0.224809\ lbf$
\end{itemize}	

\item[Frequency:] hertz ($Hz$)
\begin{itemize}
\items $=s^{-1}$
\end{itemize}				

\item[Illuminance:] lux ($lx$)
\begin{itemize}
\items $= lm\cdot m^{2} \tab[0.5cm]= m^{-2} \cdot cd$
\end{itemize}							

\item[Luminous flux:] lumen ($lm$)
\begin{itemize}
\items $= cd \cdot sr \tab[0.5cm]= cd$
\end{itemize}				

\item[Magnetic flux:] weber ($Wb$)
\begin{itemize}
\items $= V \cdot s \tab[0.5cm]= kg \cdot m^2 \cdot s^{-2} \cdot A^{-1}$
\end{itemize}				

\item[Magnetic flux density:] tesla ($T$)
\begin{itemize}
\items $= kg \cdot s^{-2} \cdot A^{-1}$
\end{itemize}				

\item[Power:] watt ($W$)
\begin{itemize}
\items $= J \cdot s \tab[0.5cm]= kg \cdot m^2 \cdot s^{-3}$
\end{itemize}

\item[Pressure:] pascal ($Pa$)
\begin{itemize}
\items $= N\cdot m^{-2} \tab[0.5cm]= kg \cdot m^{-1} \cdot s^{-2}$
\end{itemize}

\item[Radioactivity:] becquerel ($\Omega$)
\begin{itemize}
\items Decays per second
\items $= s^{-1}$
\end{itemize}
                
\item[Solid angle:] steradian ($sr$)
\begin{itemize}
\items $ = m^2\cdot m^{-2}$
\end{itemize}

\item[Temperature:] degree Celcius ($\degrees C$)
\begin{itemize}
\items $T [C] = T [K] - 273.15$
\end{itemize}					
\end{description}			
						
\section{CGS (centimetres-grams-seconds)}

Commonly used in astronomy, to everyone's disappointment.
\begin{description}

\item[Acceleration:] gal ($Gal$)
\begin{itemize}
\items $=cm\cdot s^{-2} \tab[0.5cm]=10^-2\  m \cdot s^{-2}$
\end{itemize}

\item[Energy:] erg ($erg$)
\begin{itemize}
\items $=g\cdot cm^{2}\cdot s^{-2} \tab[0.5cm]=10^{-7}\ J$
\end{itemize}

\item[Force:] dyne ($dyn$)
\begin{itemize}
\items $=g\cdot cm \cdot s^{-2} \tab[0.5cm]=10^{-5}\ N$
\end{itemize}

\item[Length:] centimetre ($cm$)
\begin{itemize}
\items $= 0.01\ m$
\end{itemize}				

\item[Mass:] gram ($g$)
\begin{itemize}
\items $=10^{-3}\  kg$
\end{itemize}

\item[Power:] erg per second ($erg/s$)
\begin{itemize}
\items $=g\cdot cm^{2}\cdot s^{-2} \tab[0.5cm]= 10^{-7}\ W$
\end{itemize}

\item[Pressure:] barye ($Ba$)
\begin{itemize}
\items $=g\cdot cm^{-1}\cdot s^{-2} \tab[0.5cm]= 10^{-1}\ Pa$
\end{itemize}
                
\item[Time:] second ($s$)

\item[Velocity:] centimetre per second ($cm/s$)
\begin{itemize}
\items $= 10^{-2}\ m\cdot s^{-1}$
\end{itemize}

\item[Viscosity (dynamic):] poise ($P$)
\begin{itemize}
\items $=g\cdot cm^{-1}\ s^{-1} \tab[0.5cm]=10^{-1}\ Pa\cdot s$
\end{itemize}

\item[Viscosity (kinematic):] stokes ($St$)
\begin{itemize}
\items $=g\cdot cm^{2}\ s^{-1} \tab[0.5cm]=10^{-4}\ m^2 \cdot s^{-1}$
\end{itemize}

\item[Wavenumber:] kayser ($K$)
\begin{itemize}
\items $=cm^{-1} \tab[0.5cm]\tab[0.5cm]=100\ m^{-1}$
\end{itemize}				
\end{description}
			

\section{Natural Units}	
Handy when you're dealing with small things.

\begin{description}				

\item[Charge:] elementary charge ($e$)
\begin{itemize}
\items The electric charge of a proton.
\items $= 1.602\ 176\ 565 \times 10^{-19}\ C \tab[0.5cm]\approx 1.6 \times 10^{-19}\ C$
\end{itemize}

\item[Energy:] electron volt ($eV$)
\begin{itemize}
\item The work done to move an electron across one volt of potential.
\items $= e\cdot V = 1.602\ 176\ 565 \times 10^{-19}\ J \tab[0.5cm]\approx 1.6 \times 10^{-19}\ J$
\end{itemize}					
\end{description}	
            

\section{Planck Units}
Units based around the five universal constants $c$, $G$, $\hbar$, $k_e = \dfrac{1}{4\pi\varepsilon_0}$ and $k_B$. Only the base units are listed; units for other quantities can be easily derived from these.

\begin{description}

\item[Planck length ($l_P$):] \( =\sqrt[]{\dfrac{\hbar G}{c^3}} \tab[0.5cm] = 1.616\ 229 \times 10^{-35} m\)

\item[Planck mass ($l_P$):] \( = \sqrt[]{\dfrac{\hbar c}{G}} \tab[0.5cm] = 2.176\ 470 \times 10^{-8} kg\)

\item[Planck time ($l_P$):] \( = \dfrac{l_P}{c} \tab[0.5cm] = \dfrac{\hbar}{m_P c^2} \tab[0.5cm] = \sqrt[]{\dfrac{\hbar G}{c^5}} \tab[0.5cm] = 5.391\ 16 \times 10^{-44} s\)

\item[Planck charge ($q_P$):] \( = \sqrt[]{4\pi\varepsilon_0\hbar c} \tab[0.5cm] = \dfrac{e}{\sqrt[]{\alpha}}\tab[0.5cm] = 1.875\ 545\ 956 \times 10^{-18} C\)

\item[Planck temperature ($T_P$):] \( = \dfrac{m_P c^2}{k_B} \tab[0.5cm] = \sqrt[]{\dfrac{\hbar c^5}{Gk_B^2}} \tab[0.5cm] = 1.416\ 808 \times 10^{32} K\)

\end{description}

            
\section{Astronomy units}


\subsection{Astronomical system}
\begin{description}

\item[Distance:] astronomical unit ($AU$)
\begin{itemize}
\items Roughly the distance from the Earth to the Sun.
\items $= 1.4960 \times 10^{11}\ m \tab[0.5cm]= 4.8481 \times 10^{-6}\ pc \tab[0.5cm]= 1.5813 \times 10^{-5}\ ly$
\end{itemize}

\item[Mass:] solar mass ($M_\odot$)
\begin{itemize}
\items $= 1.98855 \times 10^{30}\ kg\approx 2 \times 10^{30}\ kg = 1048\ M_{\jupiter} = 332\ 950\ M_\odot$
\end{itemize}

\item[Time:] Day
\begin{itemize}
\items $= 86\ 400 s$
\end{itemize}
\end{description}

\subsubsection{Complimentary units}

\begin{description}
\item[Distance:] Solar radius ($R_\odot$)
\begin{itemize}
\items $=6.957 \times 10^8\ m \tab[0.5cm]= 695\ 700\ km \tab[0.5cm] \approx 7 \times 10^8\ m$	
\end{itemize}

\item[Distance:] parsec ($pc$)
\begin{itemize}
\items The distance at which the parallax of an object over the course of the Earth's orbit is one arcsec.
\items $= 3.0857 \times 10^{16}\ m \tab[0.5cm]= 2.0626 \times 10^5\ AU \tab[0.5cm]= 3.26156\ ly$
\end{itemize}

\item[Distance:] light year ($ly$)
\begin{itemize}
\items The distance travelled by light in a vacuum in a year.
\items $= 9.4607 \times 10^{15}\ m \tab[0.5cm] = 6.3241 \times 10^4\ AU \tab[0.5cm]= 0.3066\ pc$
\end{itemize}

\item[Mass:] Earth mass ($M_{\oplus}$)
\begin{itemize}
\items $= 5.9722 \times 10^{24} kg \tab[0.5cm] \approx 6 \times 10^{27} kg $
\end{itemize}

\item[Mass:] Jupiter mass ($M_{\jupiter}$)
\begin{itemize}
\items $= 1.898 \times 10^{27} kg \tab[0.5cm] \approx 1.9 \times 10^{27} kg $
\end{itemize}
                
\item[Specific Flux:] Jansky ($Jy$)
\begin{itemize}
\items $= 10^{-26}\ W\cdot m^{-2} \cdot Hz^{-1}$
\end{itemize}
\end{description}

\subsection{Equatorial Coordinate System}

\subsubsection{Right Ascension ($\alpha$)}

\begin{description}
\item[Hour ($^h$):] \( \dfrac{1}{24}\ circle \tab[0.5cm]= 15 \degrees \)
\item[Minute ($^m$):] \( \dfrac{1}{60} ^h \tab[0.5cm]= \dfrac{1}{1440}\ circle \tab[0.5cm]= 15' \)
\item[Second ($^s$):] \( \dfrac{1}{60} ^m \tab[0.5cm]= \dfrac{1}{3600} ^h \tab[0.5cm]= \dfrac{1}{86400}\ circle \tab[0.5cm]= 15" \) 

\end{description}

\subsubsection{Declination ($\delta$)}
Declination is measured using normal degrees (see \textit{Degrees of Angle}) from the equator.


\section{United States customary units (aka Imperial Units)}

\subsection{Length}
\begin{description}
\item[Point ($p$):]$ = \dfrac{127}{360}\ mm$

\item[Pica ($P/$):]$ = 12\ p \tab[0.5cm]= \dfrac{127}{30}\ mm$

\item[Inch ($in$ or $"$):]$ = 6\ P/  \tab[0.5cm]= 25.4\ mm$

\item[Foot ($ft$ or $'$):]$ = 12\ in \tab[0.5cm]= 0.3048\ m$

\item[Yard ($yd$):]$ = 3\ ft \tab[0.5cm]= 0.9144\ m$

\item[Mile ($Mi$):]$ = 5280\ ft \tab[0.5cm] = 1760\ yd \tab[0.5cm] = 1.609344\ km $
\end{description}

			
\section{Degrees of Angle}
\begin{description}

\item[Degree ($\degrees$):] \( \dfrac{1}{360}\ circle \tab[0.5cm]= \dfrac{\pi}{180}\ rad \tab[0.5cm] \approx 0.0174532925199433\ rad \)

\item[Minute of arc ($arcmin$ or $'$):] \( \dfrac{1}{60}\degrees \tab[0.5cm] =\dfrac{1}{21600}\ circle \tab[0.5cm] =\dfrac{\pi}{10800}\ rad \)

\item[Second of arc($arcsec$ or $"$):] $\dfrac{1}{60}\ arcmin \tab[0.5cm]=\dfrac{1}{3600} \degrees\tab[0.5cm]=\dfrac{1}{206265}\ circle \tab[0.5cm]=\dfrac{\pi}{648000}\  rad$
\end{description}
            
\section{Miscellaneous Units}

\begin{description}
\item[Area:] barn
\begin{itemize}
\items $= 100 fm^2 \tab[0.5cm] = 1\times 10^{-29}m^2$
\end{itemize}
\end{description}

\subsection{Pressure}
\begin{description}

\item[Bar ($bar$):] $ =10^5\ Pa \tab[0.5cm] \approx 0.9869\ atm $

\item[Atmosphere ($atm$):] $ =101325\ Pa \tab[0.5cm] = 1.01325\ bar $

\item[Torr ($torr$):] $ = \dfrac{1}{760}\ atm \tab[0.5cm]= \dfrac{101325}{760}\ Pa \tab[0.5cm]\approx 133.3224\ Pa $
\end{description}
			
\section{Prefixes}

\begin{description}
\item[atto ($a$) =]$\times 10^{-18}$
\item[femto ($f$) =]$\times 10^{-15}$
\item[pico ($p$) =]$\times 10^{-12}$
\item[nano ($n$) =]$\times 10^{-9}$
\item[micro ($\mu$) =]$\times 10^{-6}$
\item[milli ($m$) =]$\times 10^{-3}$
\item[centi ($c$) =]$\times 10^{-2}$
\item[deca ($da$) =]$\times 10^1$
\item[hecto ($h$) =]$\times 10^2$
\item[kilo ($k$) =]$\times 10^3$
\item[mega ($M$) =]$\times 10^6$
\item[giga ($G$) =]$\times 10^9$
\item[tera ($T$) =]$\times 10^{12}$
\item[peta ($P$) =]$\times 10^{15}$
\item[exa ($E$) =]$\times 10^{18}$
\end{description}