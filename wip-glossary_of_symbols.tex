\chapter{Glossary of Symbols}
	
		\begin{description}		
        	\item $\vec{x}$ : A pronumeral with an arrow above it represents a vector quantity. 		
			\item $x$ : When that same pronumeral is seen without the arrow, it represents the \textit{magnitude} of that vector.	
			\item $\hat{x}$ : A hat indicates the \textit{unit vector}; that is, a vector with magnitude 1. When replacing the arrow on a previously used vector, it indicates that it points in the direction of that vector.	
			\item $\bar{x}$ : A bar indicates that this is the average value over several items.
			\item $\dot{x}:$ A dot indicates the first derivative of the value.		
			\item $\ddot{x}:$ Two dots is the second derivative, and so on.
		
		These things stack.
        
		\end{description}
		
		
		
	
		\begin{itemize}
			
				\item $\iff$
				\begin{itemize}
					\item\textbf{Maths:} If and only if
				\end{itemize}
			
				\item $\parallel$
				\begin{itemize}
					\item \textbf{Maths:} parallel, parallel to
				\end{itemize}
				
				\item $\perp$
				\begin{itemize}
					\item \textbf{Maths:} perpendicular, perpendicular to
				\end{itemize}
				
				\item $\odot$
				\begin{itemize}
					\item \textbf{Astronomy:} (as a subscript) pertaining to the Sun.
				\end{itemize}
				
				\item $*$
				\begin{itemize}
					\item \textbf{Maths:} (as a superscript) complex conjugate
				\end{itemize}
				
				\item $\dagger$
				\begin{itemize}
					\item \textbf{Maths:} (as a superscript) Hermitian adjoint of a matrix
				\end{itemize}
				
				\item $-1$
				\begin{itemize}
					\item \textbf{Maths:} (as a superscript) inverse function, or to the power of -1; these two are often confused and must be separated based on context.
				\end{itemize}
			
				\item \textit{A}
				\begin{itemize}
					\item\textbf{Maths:} area
					\item\textbf{Physics:} area, especially cross-sectional
				\end{itemize}
				
				\item \textit{a}
				\begin{itemize}
					\item\textbf{Maths (Geometry):} The length of the long axis of an ellipse.
					\item\textbf{Physics:} acceleration
					\begin{itemize}
						\item arbitrary position (accompanied by b)
					\end{itemize}
				\end{itemize}
				
				\item \textit{B}
				\begin{itemize}
					\item\textbf{Physics:} magnetic field
				\end{itemize}
				
				\item $b$
				\begin{itemize}
					\item\textbf{Physics:} arbitrary position (accompanied by a)
					\item\textbf{Maths (Geometry):} The length of the shortest axis of an ellipse.
				\end{itemize}
				
				\item $C$
				\begin{itemize}
					\item\textbf{Unit:} Coulomb (charge)
				\end{itemize}
				
				\item $c$
				\begin{itemize}
					\item\textbf{Physics:} speed of light
					\begin{itemize}
						\item (as a subscript) centripetal
					\end{itemize}
				\end{itemize}
				
				\item $D$
				\begin{itemize}
					\item\textbf{Maths:} in linear algebra, the diagonalisation of a matrix; has the eigenvalues as its entries.
					\item\textbf{Astronomy (instrumentation):} Lens diameter
				\end{itemize}
				
				\item $D_T$
				\begin{itemize}
					\item Telescope diameter
				\end{itemize}
				
				\item $d$
				\begin{itemize}
					\item\textbf{Maths (Calculus):} preceding a quantity, it represents an infinitesemal difference in that quantity.
					\item\textbf{Maths (Linear Algebra):} The dimension of a space
					\item\textbf{Physics:} (as subscript) displaced
					
				\end{itemize}
				
				\item \textit{E}
				\begin{itemize}
					\item\textbf{Physics:} energy
					\item\textbf{Astronomy:} (as a subscript) Pertaining to the Earth
				\end{itemize}
				
				\item \textit{e}
				\begin{itemize}
					\item\textbf{Physics:} the elementary charge, charge of a proton
				\end{itemize}
				
				\item \textit{F}
				\begin{itemize}
					\item\textbf{Physics:} force, net force
					\item\textbf{Astronomy:} flux
				\end{itemize}
				
				\item $F_0$
				\begin{itemize}
					\item\textbf{Astronomy:} reference flux
				\end{itemize}
				
				\item $f$
				\begin{itemize}
					\item\textbf{Physics:} frequency (also $nu$)
					\item\textbf{Physics (optical):} Focal length
				\end{itemize}
				
				\item $f_{sys}$
				\begin{itemize}
					\item\textbf{Astronomy (instrumentation):} Focal length of the system.
				\end{itemize}
				
				\item $f_\#$
				\begin{itemize}
					\item\textbf{Astronomy (instrumentation):} Focal ratio or focal number
				\end{itemize}
				
				\item $G$
				\begin{itemize}
					\item\textbf{Physics:} universal gravitational constant
					\begin{itemize}
						\item (as a subscript) to do with gravity
					\end{itemize}
				\end{itemize}
				
				\item \textit{g}
				\begin{itemize}
					\item\textbf{Unit:} gram
					\item\textbf{Physics:} acceleration due to gravity; usually 9.8 $ms^{-2}$
				\end{itemize}
				
				\item \textit{H}
				\begin{itemize}
					\item\textbf{Maths:} a Hermitian matrix
					\item\textbf{Chemistry:} Hydrogen
					\item\textbf{Astronomy:} Mass of hydrogen in an object
				\end{itemize}
				
				\item $h$
				\begin{itemize}
					\item\textbf{Physics:} height
					\item\textbf{Astronomy:} height of the galaxy at a given point.
				\end{itemize}
				
				\item $I$
				\begin{itemize}
					\item\textbf{Physics:} electrical current
					\item\textbf{Maths:} The identity matrix
				\end{itemize}
				
				\item $i$
				\begin{itemize}
					\item \textbf{Maths:} imaginary unit, $\sqrt{-1}$
					\item\textbf{Physics:} electrical current per unit charge
				\end{itemize}
				
				\item $j$
				\begin{itemize}
					\item\textbf{Maths (linear algebra):} The row of a matrix entry
				\end{itemize}
				
				\item $k$
				\begin{itemize}
					\item\textbf{Maths (linear algebra):} The column of a matrix entry
				\end{itemize}
				
				\item $k_B$
				\begin{itemize}
					\item\textbf{Physics:} Boltzmann constant
					\item
				\end{itemize}
				
				\item $k_e$
				\begin{itemize}
					\item\textbf{Physics (Electromagnetism):} Coulomb's constant
					\item
				\end{itemize}
				
				\item $k_s$
				\begin{itemize}
					\item\textbf{Physics:} Spring stiffness constant
					\item
				\end{itemize}
				
				\item $L$
				\begin{itemize}
					\item\textbf{Physics:} length
					\begin{itemize}
						\item 
					\end{itemize}
				\end{itemize}
				
				\item $l$
				\begin{itemize}
					\item\textbf{Physics:} length
					\begin{itemize}
						\item path
					\end{itemize}
				\end{itemize}
				
				\item $\ell$
				\begin{itemize}
					\item\textbf{Physics:} Lagrangian
				\end{itemize}
				
				\item $M$
				\begin{itemize}
					\item\textbf{Physics:} mass, especially the larger of two or as a constant
					\item\textbf{Astronomy:} absolute magnitude
				\end{itemize}
				
				\item $m$
				\begin{itemize}
					\item\textbf{Unit:} metre (length)
					\item\textbf{Maths:} the number of rows in a matrix
					\item\textbf{Physics:} mass, especially the smaller of two or as a variable
					\item\textbf{Astronomy:} apparent magnitude
				\end{itemize}
				
				\item $m_0$
				\begin{itemize}
					\item \textbf{Astronomy:} reference magnitude
				\end{itemize}
				
				\item $n$
				\begin{itemize}
					\item\textbf{Maths:} the number of columns in a matrix
					\begin{itemize}
						\item the number of rows and the number of columns in a square matrix
					\end{itemize}
					\item\textbf{Physics:} number density; ie, the number per unit volume or area
				\end{itemize}
				
				\item $N$
				\begin{itemize}
					\item\textbf{Astronomy (instrumentation):} Focal ratio
				\end{itemize}
				
				\item $P$
				\begin{itemize}
					\item\textbf{Maths:} the eigenvector matrix
					\begin{itemize}
						\item a projector
						\item probability
					\end{itemize}
					\item\textbf{Physics:} power
					\begin{itemize}
                    	\item pressure
						\item probability
					\end{itemize}
				\end{itemize}
				
				\item $p$
				\begin{itemize}
					\item\textbf{Physics:} momentum
                    \begin{itemize}
                    	\item pressure
                    \end{itemize}
					\item\textbf{Astronomy (Instrumentation):} Pixel length
				\end{itemize}
				
				\item \textit{R}
				\begin{itemize}
					\item\textbf{Maths:} radius, especially fixed
					\item\textbf{Physics:} resistance
					\item\textbf{Astronomy (Instrumentation):} The radius of curvature of a lens
				\end{itemize}
				
				\item $R_1$
				\begin{itemize}
					\item\textbf{Astronomy (Instrumentation):} The radius of curvature of the object-facing side of a lens.
				\end{itemize}
				
				\item $R_2$
				\begin{itemize}
					\item\textbf{Astronomy (Instrumentation):} The radius of curvature of the image-facing side of a lens.
				\end{itemize}
				
				\item $R_S$
				\begin{itemize}
					\item \textbf{Physics (Relativity):}Schwarszchild radius
				\end{itemize}
				
				\item $r$
				\begin{itemize}
					\item\textbf{Maths:} radius, especially as a variable
					\begin{itemize}
						\item distance
						\item distance from origin in polar or spherical coordinates
						\item distance from the z-axis in cylindrical coordinates (also $\rho$)
					\end{itemize}
					\item\textbf{Physics:} position
					\begin{itemize}
						\item radius
						\item distance 
					\end{itemize}
				\end{itemize}
				
				\item$r_0$
				\begin{itemize}
					\item \textbf{Astronomy (Instrumentation):} The Fried parameter, or atmospheric coherence length, used as a measurement of atmospheric turbulence. 
				\end{itemize}
								
				\item$r'$
				\begin{itemize}
					\item \textbf{Physics:} position of the source of a field
				\end{itemize}
				
				\item \textit{S}
				\begin{itemize}
					\item\textbf{Physics (Relativity):} A frame of reference
					\begin{itemize}
						\item (as a subscript) Schwarszchild
					\end{itemize}
				\end{itemize}
				
				\item \textit{s}
				\begin{itemize}
					\item\textbf{Unit:} second
					\item\textbf{Physics:} separation
					\begin{itemize}
						\item (as a subscript) Relating to springs.
					\end{itemize}
				\end{itemize}
				
				\item \textit{T}
				\begin{itemize}
					\item \textbf{Maths:} (as a superscript) transpose of a matrix
					\item\textbf{Physics:} period
					\begin{itemize}
						\item temperature                      
					\end{itemize}
				\end{itemize}
				
				\item \textit{t}
				\begin{itemize}
					\item\textbf{Physics:} time
				\end{itemize}
				
				\item \textit{U}
				\begin{itemize}
					\item\textbf{Physics:} potential energy
					\item\textbf{Maths:} a unitary matrix
				\end{itemize}
				
				\item \textit{u}
				\begin{itemize}
					\item\textbf{Physics:} initial velocity
				\end{itemize}
				
				\item \textit{V}
				\begin{itemize}
					\item\textbf{Physics:} voltage
					\begin{itemize}
						\item electric potential
						\item volume
					\end{itemize}
				\end{itemize}
				
				\item $v$
				\begin{itemize}
					\item\textbf{Physics:} velocity
				\end{itemize}
				
				\item $w$
				\begin{itemize}
					\item\textbf{Maths:} width
				\end{itemize}
				
				\item $w_D$
				\begin{itemize}
					\item\textbf{Astronomy (instrumentation):} width of detector
				\end{itemize}
				
				\item $X$
				\begin{itemize}
					\item\textbf{Physics:} Pauli operator
					\begin{itemize}
						\item 
					\end{itemize}
				\end{itemize}
				
				\item \textit{x}
				\begin{itemize}
					\item\textbf{Maths:} a variable, especially in a function.
					\begin{itemize}
						\item the horizontal axis in Cartesian coordinates.
					\end{itemize}
					\item \textbf{Physics:} position
				\end{itemize}
				
				\item $x_0$
				\begin{itemize}
					\item\textbf{Maths:} the initial value of variable $x$.
				\end{itemize}
				
				\item \textit{Y}
				\begin{itemize}
					\item\textbf{Physics:} A Pauli operator
					\begin{itemize}
						\item 
					\end{itemize}
				\end{itemize}
				
				\item \textit{y}
				\begin{itemize}
					\item\textbf{Maths:} an axis in Cartesian coordinates.
					\item \textbf{Physics:} position, usually vertical.
				\end{itemize}
				
				\item \textit{Z}
				\begin{itemize}
					\item\textbf{Physics:} A Pauli operator
					\begin{itemize}
						\item 
					\end{itemize}
				\end{itemize}
				
				\item \textit{z}
				\begin{itemize}
					\item\textbf{Maths:} the vertical axis in Cartesian coordinates.
					\begin{itemize}
						\item A complex number
					\end{itemize}
					\item \textbf{Physics:} position
				\end{itemize}
				
				\item $z_0$
				\begin{itemize}
					\item\textbf{Maths:} the initial value of variable $z$.
				\end{itemize}
								
				\item $\alpha$
				
				\item $\beta$
				
				\item $\Gamma$
				
				\item $\gamma$
				\begin{itemize}
					\item \textbf{Physics:} gamma factor
				\end{itemize}
				
				\item $\Delta$
				\begin{itemize}
					\item \textbf{Maths:} difference between two quantities
					\item \textbf{Physics:} difference between two quantities
					\begin{itemize}
						\item \textbf{Standard deviation}
					\end{itemize}
				\end{itemize}
				
				\item $\delta$
				\begin{itemize}
					\item 
				\end{itemize}
				
				\item $\partial$
				\begin{itemize}
					\item \item\textbf{Maths (Calculus):} denotes the patial derivative
				\end{itemize}
								
				\item $\epsilon$
				\begin{itemize}
					\item 
				\end{itemize}	
								
				\item $\varepsilon$
				\begin{itemize}
					\item \textbf{Physics:} electromotive force (force per unit charge)
				\end{itemize}
				
				\item $\varepsilon_d$
				\begin{itemize}
					\item \textbf{Astronomy (instrumentation):} Diffraction limit to angular resolution.
				\end{itemize}
				
				\item $\varepsilon_s$
				\begin{itemize}
					\item \textbf{Astronomy (instrumentation):} Seeing limit to angular resolution.
				\end{itemize}
				
				\item $\zeta$
				\begin{itemize}
					\item 
				\end{itemize}
				
				\item $\eta$
				\begin{itemize}
					\item 
				\end{itemize}
				
				\item $\Theta$
				\begin{itemize}
					\item 
				\end{itemize}
				
				\item $\theta$
				\begin{itemize}
					\item \textbf{Maths:} An angle
				\end{itemize}
				
				\item $\vartheta$
				\begin{itemize}
					\item 
				\end{itemize}
				
				\item $\iota$
				\begin{itemize}
					\item 
				\end{itemize}
				
				\item $\kappa$
				\begin{itemize}
					\item 
				\end{itemize}
				
				\item $\Lambda$
				\begin{itemize}
					\item 
				\end{itemize}
				
				\item $\lambda$
				\begin{itemize}
					\item \textbf{Maths:} eigenvalue
					\item \textbf{Physics:} wavelength
				\end{itemize}
				
				\item $\mu$
				\begin{itemize}
					\item \textbf{Physics:} mass density
				\end{itemize}
				
				\item $\nu$
				\begin{itemize}
					\item \textbf{Physics:} frequency (also $f$)
				\end{itemize}
				
				\item $\Xi$
				\begin{itemize}
					\item 
				\end{itemize}
				
				\item $\xi$
				\begin{itemize}
					\item 
				\end{itemize}
				
				\item $\Pi$
				\begin{itemize}
					\item 
				\end{itemize}
				
				\item $\pi$
				\begin{itemize}
					\item\textbf{Maths:} the ratio of a circle's circumference to its diameter.
				\end{itemize}
				
				\item $\rho$
				\begin{itemize}
					\item \textbf{Physics:} density
					\begin{itemize}
						\item resistivity
					\end{itemize}
				\end{itemize}
				
				\item $\varrho$
				\begin{itemize}
					\item 
				\end{itemize}
				
				\item $\Sigma$
				\begin{itemize}
					\item 
				\end{itemize}
				
				\item $\sigma$
				\begin{itemize}
					\item \textbf{Physics:} conductivity
					\begin{itemize}
						\item Pauli operator
					\end{itemize}
					\item\textbf{Astronomy (Instrumentation:)} instrument error
				\end{itemize}
				
				\item $\sigma_{fit}$
				\begin{itemize}
					\item\textbf{Astronomy (Instrumentation:)} fitting error
				\end{itemize}
				
				\item $\tau$
				\begin{itemize}
					\item \textbf{Physics:} in relativity, proper time
				\end{itemize}
				
				\item $\upsilon$
				\begin{itemize}
					\item 
				\end{itemize}
				
				\item $\Upsilon$
				\begin{itemize}
					\item 
				\end{itemize}
				
				\item $\Phi$
				\begin{itemize}
					\item \textbf{Physics:} Gravitational potential (GPE per unit mass)
				\end{itemize}
				
				\item $\phi$
				\begin{itemize}
					\item 
				\end{itemize}
				
				\item $\varphi$
				\begin{itemize}
					\item 
				\end{itemize}
				
				\item $\Psi$
				\begin{itemize}
					\item 
				\end{itemize}
				
				\item $\psi$
				\begin{itemize}
					\item \textbf{Physics (Quantum):} a quantum state
				\end{itemize}
				
				\item $\omega$
				\begin{itemize}
					\item 
				\end{itemize}
				
				\item $\Omega$
				\begin{itemize}
					\item 
				\end{itemize}
				
				
		\end{itemize}